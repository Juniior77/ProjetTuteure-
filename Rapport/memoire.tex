\documentclass[a4paper, 12pt]{book}
\usepackage{graphicx}
\usepackage[french]{babel}
\usepackage[utf8]{inputenc}
\usepackage[T1]{fontenc}
\usepackage{multirow}
\usepackage{listings}
\usepackage{float}
\usepackage{url}
\usepackage[french]{algorithm}
\usepackage{style/myalgorithm}
\usepackage{amsmath,amsfonts,amssymb}
\newcommand{\fBm}{\emph{fBm}~}
\newcommand{\etal}{\emph{et al.}~}
\newcommand{\glAd}{\emph{GL4D}~}
\newcommand{\apiopengl}{API OpenGL\textsuperscript{\textregistered}~}
\newcommand{\opengl}{OpenGL\textsuperscript{\textregistered}~}
\newcommand{\opengles}{OpenGL\textsuperscript{\textregistered}ES~}
\newcommand{\clang}{langage \texttt{C}}
\newcommand{\codesource}{\textsc{Code source}~}
\floatstyle{ruled}
\newfloat{programslist}{htbp}{locs}
\newcommand{\listofprograms}{\listof{programslist}{Liste des codes source}}
\newcounter{program}[subsection]
\renewcommand{\theprogram}{\arabic{chapter}.\arabic{program}}

\newenvironment{program}[1]{
  \if\relax\detokenize{#1}\relax
  \gdef\mycaption{\relax}
  \else
  \gdef\mycaption{#1}
  \fi
  \refstepcounter{program}
  \addcontentsline{locs}{section}{#1}
  \footnotesize
}{
  \begin{description}
    \item[\codesource \theprogram]--~\mycaption
  \end{description}
}

\begin{document}
\begin{titlepage}
  \begin{center}
    \begin{tabular*}{\textwidth}{l@{\extracolsep{\fill}}r}
      \includegraphics[height=1.5cm]{images/m1info.png}
    \end{tabular*}
    \small 
    \rule{\textwidth}{.5pt}~\\
    \large 
    \textsc{Université Paris 8 - Vincennes à Saint-Denis}\vspace{0.5cm}\\
    \textbf{Master Informatique des Systèmes Embarqués}\vspace{3.0cm}\\
    \Large
    \textbf{projet tuteuré : Véhicule Autonome}\vspace{1.5cm}\\
    \large
    \textbf{Vilaire \textsc{Guillaume}}\vspace{1.5cm}\\
	\textbf{De Oliveira \textsc{Anthony}}\vspace{1.5cm}\\
    Date de soutenance : le 31/01/2017\vspace{1.75cm}\\
  \end{center}\vspace{1.5cm}~\\
  \begin{tabular}{ll}
    \hspace{-0.45cm}Tuteur -- Université~:~&~Adrien \textsc{Revault D'allonnes}\\
  \end{tabular}
\end{titlepage}
\frontmatter

%% Table des matières
\tableofcontents
%% La liste des figure est optionnelle (si votre rapport manque de
%% contenu ajouter ce type de pages sera perçu négativement)
\mainmatter
\chapter*{Introduction}
\markboth{\sc Introduction}{}
\addcontentsline{toc}{chapter}{Introduction}
Nous avions vu précédemment quelques éléments permettant de rendre un véhicule autonome, et les différentes contraintes liés au fait de rendre un véhicule autonome.\\

Nous partions en effet avec une première contrainte qui est le positionnement du véhicule dans l'espace, avec de prime abord une position relative dans l'espace, qui pouvait être donnée via GPS par exemple, avec des coordonnées, et la position des éléments qui entourent le véhicule, permettant au véhicule d'éviter ces mêmes éléments.
\\

Nous allons donc vous présenter ici l'avancement du projet vis-à-vis du premier état de l'art, des capteurs réellement utilisés, et non pas ceux qui pouvaient potentiellement être utilisés, et l'avancement du projet vis-à-vis de notre objectif principal.\\

Comment rendre un véhicule autonome, ou dans un premier temps, rendre un véhicule conscient des éléments qui l'entourent, et lui permettre d'assister le pilote, afin d'éviter des obstacles ?

Nous allons donc voir dans un premier temps une liste des différents capteurs que nous avons utilisés pour arriver à l'état d'avancement du véhicule aujourd'hui, puis des différents types d'algorithmes utilisés.

\chapter{Base matériel et planning de développement}
Dans ce premier chapitre, nous allons parler des différents capteurs, ou éléments matériel utilisés comme base dans notre projet et leurs utilités général dans ce projet.

\section{Capteurs et éléments du projet}
Afin de réaliser ce véhicule, nous avons donc finalement utilisé plusieurs éléments afin de permettre au moins une conduite assistée du véhicule.

Voici donc la liste des différents capteurs et éléments utilisés lors de la conception du véhicule :

Capteurs à ultrasons : ces capteurs permettront au véhicule de détecter les obstacles proches de lui-même, à un fonctionnement en quelque sorte similaires a un sonar, ou l’émission d’onde sonore (ultrasons,qui va nous permettre de savoir si un obstacle est à proximité). Il sera notamment utilisé lors de fonction, comme l'aide au stationnement, et aussi a l'évitement d'obstacles.

Détecteur de ligne : Ce type de capteurs, dont le fonctionnement est quelque peu similaire à celui d’un capteur photosensible.
 
Jeu de LED : Reproduction de l'éclairage d'un véhicule, avec reproduction phare avant/arrière, clignotant, warning, et feu de frein.

Utilisation bluetooth pour le pilotage.

A modifier

\section{L'avancement du projet}

Le projet a avancé en plusieurs parties, avec dans un premier temps le pilotage du véhicule

\chapter{Implémentation et résultat pour le projet}

Nous allons vous parler dans cette partie des différents éléments intégrés au véhicule, et expliqué donc le fonctionnement de notre implémentation.


\section{Pilotage du véhicule}


\section{Détection d'obstacles}
Utilisation 

\section{Suivi de ligne}
Pour le suivi de ligne, nous nous sommes donc appuyer sur des détecteur de lignes, qui sont donc des capteurs infrarouges avec un récepteur, qui émettent et reçoivent de la lumière depuis le sol. En fonction de la valeur retourné par le capteur, et donc de l'intensité lumineuse reçu, nous pouvons déterminer de façon flagrante 2 types de couleurs : Le blanc ou le noir.\\

 
\section{park Assist}
A ajouter ou non



\chapter{Conclusion\label{chap-conclusion}}


\bibliographystyle{alpha}
\bibliography{memoire}
\end{document}
